\documentclass{article}

\title{Review of Computer Organization and Design, The Hardware/Software Interface, RISC-V Edition}
\author{MoonPresident}

\usepackage[margin=0.8in]{geometry}

\usepackage{siunitx}
\usepackage{mathtools}

%https://en.wikibooks.org/wiki/LaTeX/List_Structures


\begin{document}
	\maketitle
	\tableofcontents
	
	\section{Computer Abstraction and Technology}
	This section outlines the main idea's underpinning digital design in recent and contemporary history, circa 2020.
		\subsection{Introduction}
			This obligatory section reminds the reader the extent to which computers have revolutionised the world, in case the audience just emerged from a secluded cave on the Isle of Wight after 50 years of ritual meditation. 
			
			The highlight of this section is an indeterminable claim that if motor cars had improved at the same rate as computers we would be able to travel from New York to London in 1 second for a penny.
			
			Beyond that the section highlights the myriad of triumphs in the computing world, noting auto-mobile computing, cell phones, the sequencing of the human genome, the World Wide Web and search engines.
			
			\subsubsection{Traditional Classes of Computing Applications and Their Characteristics}
			\begin{description}
				\item[Personal Computers] description
				\item[Servers] description
				\item[Embedded Computers] description
			\end{description}
			

		\subsection{Eight Great Ideas in Computer Architecture}
			\begin{description}
				\item[] description
			\end{description}
		\subsection{Below Your Program}
		
		\subsection{Under the Covers}
		
		\subsection{Technologies for Building Processors and Memory}
		
		\subsection{Performance}
		
		\subsection{The Power Wall}
		
		\subsection{The Sea Change: The Switch from Uniprocessors to Multiprocessors}
		
		\subsection{Real Stuff: Benchmarking the Intel Core I7}
		
		\subsection{Fallacies and Pitfalls}
		
		\subsection{Concluding Remarks}
		
		\subsection{Historical Perspective and Further Reading}
		
		\subsection{Exercises}
\end{document}